\documentclass{article}
\usepackage[utf8]{inputenc}
\usepackage[a4paper]{geometry}
\usepackage{graphicx}
    \graphicspath{ {Graphics/} }
\usepackage{movie15}
\usepackage{color}
\usepackage{wrapfig}
\usepackage{multirow}
\usepackage{tabu}
\usepackage[table]{xcolor}
\usepackage{subcaption}
\usepackage[colorlinks=true,urlcolor=red, pdfborder={0 0 0}]{hyperref}
\mathchardef\mhyphen="2D


\title{\huge Data Analysis and Interpretation\\
        \LARGE Report\\
        \large Assignment Three}
\author{Anish Kulkarni, Parth Jatakia, Toshi Parmar, Vedant Basu}
\date{3 November 2016}

\begin{document}
\newgeometry{left=3cm,bottom=0.1cm, right=3cm, top=3cm}
\maketitle
\newcommand\tab[1][1cm]{\hspace*{#1}}
    \begin{center} 
            \rule{16cm}{0.2pt}
    \end{center}
\section{Assigned Responsibilities}
    The following Responsibilities were designated to each team member:
    \begin{table}[h!]
    \centering
    {\rowcolors{4}{red!80!yellow!40}{red!70!yellow!30}
        \begin{tabu}to 0.8\textwidth { |X[c]||X[c]||X[c]| } 
        \hline
        Week & Role Assigned & Member\\
        \hline
        \hline
        & Leader & Parth Jatkia \\ 
        & Coder & PaVedant Basu \\ 
        & Website Manager & Toshi Parmar \\
        \multirow{-4}{10em}{\textbf Week Three}& Report Writer & Anish Kulkarni \\
        \hline
        \end{tabu}
        }
    \caption{Week By Week Summary of Assigned Roles}
    \label{Table:WorkAllocation}
    \end{table}
\section{Summary of solutions to the assigned problems}
    \begin{enumerate}
        \item {\textbf {Problem 1: Error regions}}
        \begin{enumerate}
            \item For each temperature $T_i$ assume the measurement to be a random variable that is Gaussian distributed about the mean $\overline{l_i} = mT_i + c $ and standard deviation $\sigma$.
            \item Assuming each measurement to be independent, the log likelihood function is given by:
            \[
                \log (L(m,c,\sigma)) = \sum_{i=1}^N \log \left( \frac{1}{\sqrt{2\pi \sigma ^2}} \exp \left(\frac{-(l_i-\overline{l_i})^2}{2 \sigma ^2} \right) \right)
            \]
            where $N$ is the number of measurements.
            \item We then minimize the negative of the log likelihood function using the function \emph{minimize} in python, to find the best fit parameters: $\hat{m},\hat{c} $ and $\hat{\sigma}$
            \[
                \hat{m} = 22.90 \textit{ mm/Kelvin}
            \]
            \[
                \hat{c} = 993.51 \textit{ mm}
            \]
            \[
                \hat{\sigma} = 26.17 \textit{ mm}
            \]
            \item The equations for the $n\mhyphen \sigma$ error region for $m$ and $c$ is given by:
            \[
                \log (L(m,c,\hat{\sigma})) = \log (L(\hat{m},\hat{c},\hat{\sigma})) - n*\frac{2}{2} \qquad 1\mhyphen \sigma \textit{ error region}
            \]
            Which can be expanded to get :
            \[
                m^2\sum T_i^2 + 2mc\sum T_i + Nc^2 - 2m\sum T_il_i - 2c \sum l_i + \sum l_i^2 = \sum (l_i - \overline{l_i})^2 + 2n\sigma ^2
            \]
            Following is a plot of $1\mhyphen \sigma$ and $2\mhyphen \sigma$ error regions:
            \begin{figure}[!htb]
                    \centering
                \includegraphics[scale = 0.5]{m,c_error_regions}
                    \caption{error regions for m and c}
                    
            \end{figure}
        \newpage{}
            \item
            Fixing $c$ to it's maximum likelihood estimate, 
            \[
                \textit{the error interval for } m = [22.78\textit{ mm/Kelvin},23.03 \textit{ mm/Kelvin}]
            \]
            Fixing $m$ to it's maximum likelihood estimate, 
            \[
                \textit{the error interval for } c = [992.79\textit{ mm},994.24 \textit{ mm}]
            \]
            
        \end{enumerate}
        \newpage{}
        \item {\textbf{Problem 2: Discover dark matter}}
        \begin{enumerate}
            \item Histograms of given data and background-only process were plotted:
            \begin{figure}[!htb]
                \fbox {\begin{minipage}{0.5\textwidth}
                    \centering
                    \includegraphics[width=\textwidth,height=\textheight,keepaspectratio]{dark_matter_data}
                    \caption{Recoil Energy Data}
                    \label{fig:RecoilEnergyData}
                \end{minipage}%
                }
                \fbox{\begin{minipage}{0.5\textwidth}
                    \centering
                    \includegraphics[width=\textwidth,height=\textheight,keepaspectratio]{background_only}
                    \caption{Background}
                    \label{fig:Background}
                \end{minipage}%
                }
                \end{figure}
            \item Histogram of the mean number of expected events, considering signal + background was plotted for various cross-section values:
            \begin{figure}[!htb]
                \fbox {\begin{minipage}{0.4\textwidth}
                    \centering
                    \includegraphics[width=\textwidth,height=\textheight,keepaspectratio]{crossection_0,01}
                    \caption{cross-section=0.01}
                \end{minipage}%
                }
                \fbox {\begin{minipage}{0.4\textwidth}
                    \centering
                    \includegraphics[width=\textwidth,height=\textheight,keepaspectratio]{crossection_0,1}
                    \caption{cross-section=0.1}
                \end{minipage}%
                }
                \fbox {\begin{minipage}{0.4\textwidth}
                    \centering
                    \includegraphics[width=\textwidth,height=\textheight,keepaspectratio]{crossection_1}
                    \caption{cross-section=1}
                \end{minipage}%
                }
                \fbox {\begin{minipage}{0.4\textwidth}
                    \centering
                    \includegraphics[width=\textwidth,height=\textheight,keepaspectratio]{crossection_10}
                    \caption{cross-section=10}
                \end{minipage}%
                }
                \fbox {\begin{minipage}{0.4\textwidth}
                    \centering
                    \includegraphics[width=\textwidth,height=\textheight,keepaspectratio]{crossection_100}
                    \caption{cross-section=100}
                \end{minipage}%
                }
                \end{figure}
            For cross-section = 1, 10 and 100 one can tell by eye whether or not a dark matter signal is present.
            \item
            For a large number of scattering events and sufficiently small bin size, probability of measured recoil energy being in a particular bin is small. Hence, we can assume that the number of events in a particular bin is Poisson distributed with mean:
            \[
                \lambda_i = \left( \left( \frac{dN}{dE_R}\right)_{signal} + \left( \frac{dN}{dE_R}\right)_{background}\right)*1 KeV
            \]
            Further, for large number of bins, number of measurements in each bin can be considered to be independent random variables. Hence, the log likelihood function is the sum of individual log likelihood functions for each bin. Let $B$ be the number of bins and let $d_i$ and $\lambda_i$ be the number of events observed and expected, respectively, in the $i^{th}$ bin. Then,
            \[
                \log (L(\sigma)) = \sum_{i=1}^{B} \log \left( \frac{\lambda_i(\sigma)^{d_i}}{d_i!} e^{-\lambda_i(\sigma)}\right)
            \]
            which on simplification and after ignoring constants gives:
            \[
                \log (L(\sigma)) = \sum_{i=1}^{B} d_i\log (\lambda_i(\sigma)) - \lambda_i(\sigma)
            \]
            Following is a plot of the log likelihood function:
            \begin{figure}[!htb]
            \centering
                    \includegraphics[scale = 0.5]{loglikelihood}
                    \caption{Log likelihood}
            \end{figure}
            
        \item
        Maximizing the above calculated log likelihood function gives the maximum likelihood estimate (MLE) for the cross-section: 
        \[
            \hat{\sigma} = 0.17666016
        \]
        The $1\mhyphen \sigma$ interval for cross-section is given by 
        \[
            \log (L(\sigma)) = \log(L(\hat{\sigma})) -\frac{1}{2}
        \]
        \[
            1\mhyphen \sigma \textit{ interval }= [0.14699, 0.20667]
        \]
        \end{enumerate}
        
    \end{enumerate}
    
\end{document}
